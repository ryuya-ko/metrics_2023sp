\documentclass[11pt]{article}

\usepackage{import}
\import{../}{preamble}

\begin{document}

\title{Econometrics II HW3} % chktex 13
\author[1]{Harshit Kapur}
\author[2]{Ryuya Ko}
\affil[1]{UT Eid:}
\affil[2]{UT Eid: gl23453}
\date{\today}

\maketitle

\pagebreak

\section*{1}

\subsection*{a}

\subsection*{b}

\subsection*{c}
\section*{2}
\subsection*{a}


\section*{3}
\subsection*{a}

$d2_t$ always takes value $1$ if $t = 2$ and $0$ otherwise for all $i$. $prog_{it}$ takes value $1$ if $t = 2$ and the individual $i$ receives the program and $0$ otherwise. Two variables are correlated. Thus, if we exclude $d2_t$, the estimator of $\delta_1$ suffers from the omitted variable bias.

\subsection*{b}

Since we may have $\E[u_{it} \;\mid \; prog_{it}] = a \neq 0$, including $c_i$ is essential to avoid omitted variable bias.

\subsection*{c}

Notice that the first differencing method yields

\begin{align*}
    \Delta y_i = \theta_2 + \delta_1 prog_{i} + u_{i2} - u_{i1}
\end{align*}

where $prog_{i}$ take value $1$ if the individual $i$ receives the treatment and $0$ otherwise. The estimator is obtained by OLS procedure, which gives us the following expression.

\begin{align*}
    \hat{\delta}_1 &= \frac{\sum_{i=1}^{N} (prog_i - \bar{prog}) (\Delta y_i - \bar{\Delta y})}{\sum_{i=1}^{N} (prog_i - \bar{prog})^2} \\
    \hat{\theta}_2 &= \bar{\Delta y} - \hat{\delta}_1 \bar{prog}
\end{align*}

By construction, we have $\bar{prog} = N_T/N$ where $N_T$ is the number of individual who get the treatment. We have

\begin{align*}
    \sum_{i=1}^{N} (prog_i - \bar{prog})^2
    &= \sum_{i=1}^{N} prog_i^2 - N \bar{prog}^2 \\
    &= N_T (1 - \frac{N_T}{N}) \\
    \sum_{i=1}^{N} (prog_i - \bar{prog}) (\Delta y_i - \bar{\Delta y})
    &= \sum_{i=1}^{N} prog_i \Delta y_i - N \bar{prog} \bar{\Delta y} \\
    &= N_T \bar{\Delta y}_{treat} - N_T \left( \frac{N_T}{N} \bar{\Delta y}_{treat} + \frac{N - N_T}{N} \bar{\Delta y}_{control} \right) \\
    &= N_T \left( 1 - \frac{N_T}{N} \right) (\bar{\Delta y}_{treat} - \bar{\Delta y}_{control})
\end{align*}

Hence, we obtain $\hat{\delta}_1 = \bar{\Delta y}_{treat} - \bar{\Delta y}_{control}$. By substituting this expression, we also get

\begin{align*}
    \hat{\theta}_2
    &= \bar{\Delta y} - \hat{\delta}_1 \bar{prog} \\
    &= \frac{N_T}{N} \bar{\Delta y}_{treat} + \frac{N - N_T}{N} \bar{\Delta y}_{control} - \frac{N_T}{N} (\bar{\Delta y}_{treat} - \bar{\Delta y}_{control}) \\
    &= \bar{\Delta y}_{control}
\end{align*}

\section*{4}

\subsection*{d}

The OLS estimator suffers the omitted variable bias since it excludes the regressor representing the individual effect $c_i$. $c_i$ and $exec_{it}$ is highly likely to be correlated: in the state with more murderers, the number od executions increase. Hence, the OLS estimator of $\beta_1$ exhibits upward bias due to the correlation.

\subsection*{e}

The strict exogeneity require

\begin{align*}
    \E[u_{it} \mid exec_{is}, c_i] = 0
\end{align*}

for any $t, s$. That is, for example, the past history of execution cannot affect the future murder rate beyond the state-level fixed effects.

\end{document}
