\documentclass[11pt]{article}

\usepackage{import}
\import{../}{preamble}
\begin{document}

\title{Econometrics II HW2} % chktex 13
\author{}  % chktex 13
\date{\today}

\maketitle

\pagebreak

\section*{1}

The answer is $1 + 1 = 2$

\section*{2}
\section*{3}



\section*{4}

We assume $U$ to be a continuous random variable.

\subsection{a}\label{q:4_a}

Notice that

\begin{align*}
    \Pr(U \leq a \mid X = x)
\end{align*}

Take any vector $x\in \R^k$ such that $x\beta > 0$. Then, we have

\begin{align*}
    \Pr(Y = 1 \mid X = x)
    &= \Pr(x\beta + U \geq 0) \\
    &= \Pr(U \geq - x\beta) \\
    &= 1 - \Pr(U < - x\beta)
\end{align*}

Since $x\beta > 0$, $\Pr(- x\beta \leq U \leq 0) > 0$. Hence, we obtain

\begin{align*}
    \Pr(Y = 1 \mid X = x)
    &= 1 - \Pr(U < - x\beta \mid X = x) \\
    &= 1 - \left( \Pr(U \leq 0 \mid X = x) - \Pr(- x\beta \leq U \leq 0) \right) \\
    &= 1 - \left( \frac{1}{2} - \Pr(- x\beta \leq U \leq 0) \right) \\
    &> 1 - \frac{1}{2} = \frac{1}{2}
\end{align*}

A similar argument holds for in the case of $x\beta < 0$, which establishes the desired result.

\subsection{b}\label{q:4_b}

Notice that $B \equiv \{b_{(-1)}\}$ is a nonempty, closed, and convex set. Since $c_{(-1)} \notin B$, by Separating hyperplane theorem, we obtain the existence of some vector $p \in \R^{k-1}\setminus \{0\}$ such that

\begin{align*}
    p \cdot b_{(-1)} < p \cdot c_{(-1)}
\end{align*}

Let $x_1$ be a scalor: $x_1 = - p \cdot b_{(-1)} - 1/2 (p\cdot c_{(-1)} - p \cdot b_{(-1)})$. Then, we obtain

\begin{align*}
    x_1 + p \cdot b_{(-1)} < 0 < x_1 + p \cdot c_{(-1)}
\end{align*}

By constructing $x = (x_1, p^\prime)^\prime \in \R^k$, we obtain the desired result.

\subsection{c}

Under the two different specification $b$ and $c$, the result of part~\ref{q:4_b} implies that there exists an observation $x \in \R^k$ such that

\begin{align*}
    xb < 0 < xc
\end{align*}

For such observation, the result of part~\ref{4_a} implies

\begin{align*}
    \Pr(Y = 1 \mid X = x, \beta = b) &< \frac{1}{2} \\
    \Pr(Y = 1 \mid X = x, \beta = c) &> \frac{1}{2}
\end{align*}

Hence, the distribution of $Y$ can be different between the two specification.

\end{document}
