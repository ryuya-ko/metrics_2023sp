\documentclass[11pt]{article}

\usepackage{import}
\import{../}{preamble}
\begin{document}

\title{Econometrics II HW2} % chktex 13
\author{}  % chktex 13
\date{\today}

\maketitle

\pagebreak

\section*{1}

\subsection*{a}

The marginal effect of $x_k$ is

\begin{align*}
    \begin{cases}{}
        \beta_k & \text{LPM} \\
        \beta_k \phi(x\beta) & \text{probit} \\
        \beta_k \frac{G(x\beta)}{1 + e^{x\beta}} & \text{logit}
    \end{cases}
\end{align*}

\subsection*{b}
\subsubsection*{i and ii}

For probit model, the problem is written as

\begin{align*}
    \max_{x}&  \mid \beta_k \phi(x\beta) \mid
\end{align*}

Since $\phi > 0$ and $\phi(0)$ is the maximum value of the normal density, we have a solution $x^*$ such that $x^* \beta = 0$\footnote{We presume $\beta_k \neq 0$ to avoid a trivial argument.}. This yields the maximum value of the marginal effect as $\mid \beta_k \mid \phi(0) = \mid \beta_k \mid \frac{1}{\sqrt{2\pi}}$.

For logit model, the problem is written as

\begin{align*}
    \max_{x}&  \mid \beta_k \mid\frac{G(x\beta)}{1 + e^{x\beta}}
    = \mid \beta_k \mid \max_{x} \frac{G(x\beta)}{1 + e^{x\beta}}
\end{align*}

The solution $x^*$ satisfies the first-order condition

\begin{align*}
    0 &= \frac{\partial G(x\beta) / (1 + e^{x\beta})}{\partial x}
\end{align*}

which implies $e^{x^*\beta} = 1$, i.e., $x^*\beta = 0$.

This yields the maximum value of the marginal effect as $\mid \beta_k \mid /4$.



\subsubsection*{iii}

The marginal effect of $x_k$ at $x = 0$ is $\beta_k \phi(0) = \beta_k \frac{1}{\sqrt{2\pi}}$ and $\beta_k / 4$ under probit and logit model, respectively. The former can be approximated as $0.398 \beta_k$.


\subsection*{c}
\subsubsection*{i}

Since the marginal effect of $x_k$ in LPM is always $\beta_k$, the setting implies

\begin{align*}
    \beta_{LPM} = 0.398 \beta_{probit}
\end{align*}

\subsubsection*{ii}

\begin{align*}
    4 \beta_{LPM} = \beta_{logit}
\end{align*}

\subsubsection*{iii}

From the previous observation, we obtain

\begin{align*}
    \beta_{probit} = \frac{1}{2 \cdot 0.398}\beta_{logit}
\end{align*}


\section*{2}
\section*{3}

\subsection*{b}

\subsubsection*{ii A}

Notice that

\begin{align*}
    \bar{x}^\prime \beta &= -1.114 \\
    \bar{x}^\prime \beta + \beta_{\text{mothereduc}}&= -1.259
\end{align*}

Hence, the efect of one additional year of mother's education is

\begin{align*}
    G(-1.259) - G(-1.114)  = \Phi(-1.259) - \Phi(-1.114) = - 0.029
\end{align*}

\subsubsection*{ii B}

The marginal effect measured by the standard formula is

\begin{align*}
    \phi(-1.114) \beta_{\text{mothereduc}} = -0.031
\end{align*}

\subsubsection*{iv}

The estiamted marginal effect under liner probability model, probit model, and logit model is $-0.029, -0.031$, and $-0.029$. Those are pretty close.

\section*{4}

We assume $U$ to be a continuous random variable.

\subsection*{a}\label{q:4_a}

Notice that

\begin{align*}
    \Pr(U \leq a \mid X = x)
\end{align*}

Take any vector $x\in \R^k$ such that $x\beta > 0$. Then, we have

\begin{align*}
    \Pr(Y = 1 \mid X = x)
    &= \Pr(x\beta + U \geq 0) \\
    &= \Pr(U \geq - x\beta) \\
    &= 1 - \Pr(U < - x\beta)
\end{align*}

Since $x\beta > 0$, $\Pr(- x\beta \leq U \leq 0) > 0$. Hence, we obtain

\begin{align*}
    \Pr(Y = 1 \mid X = x)
    &= 1 - \Pr(U < - x\beta \mid X = x) \\
    &= 1 - \left( \Pr(U \leq 0 \mid X = x) - \Pr(- x\beta \leq U \leq 0) \right) \\
    &= 1 - \left( \frac{1}{2} - \Pr(- x\beta \leq U \leq 0) \right) \\
    &> 1 - \frac{1}{2} = \frac{1}{2}
\end{align*}

A similar argument holds for in the case of $x\beta < 0$, which establishes the desired result.

\subsection*{b}\label{q:4_b}

Notice that $B \equiv \{b_{(-1)}\}$ is a nonempty, closed, and convex set. Since $c_{(-1)} \notin B$, by Separating hyperplane theorem, we obtain the existence of some vector $p \in \R^{k-1}\setminus \{0\}$ such that

\begin{align*}
    p \cdot b_{(-1)} < p \cdot c_{(-1)}
\end{align*}

Let $x_1$ be a scalor: $x_1 = - p \cdot b_{(-1)} - 1/2 (p\cdot c_{(-1)} - p \cdot b_{(-1)})$. Then, we obtain

\begin{align*}
    x_1 + p \cdot b_{(-1)} < 0 < x_1 + p \cdot c_{(-1)}
\end{align*}

By constructing $x = (x_1, p^\prime)^\prime \in \R^k$, we obtain the desired result.

\subsection*{c}

Under the two different specification $b$ and $c$, the result of part~\ref{q:4_b} implies that there exists an observation $x \in \R^k$ such that

\begin{align*}
    xb < 0 < xc
\end{align*}

For such observation, the result of part~\ref{4_a} implies

\begin{align*}
    \Pr(Y = 1 \mid X = x, \beta = b) &< \frac{1}{2} \\
    \Pr(Y = 1 \mid X = x, \beta = c) &> \frac{1}{2}
\end{align*}

Hence, the distribution of $Y$ can be different between the two specification.

\end{document}
